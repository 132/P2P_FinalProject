
%%%%%%%%%%%%%%%%%%%%%%% file typeinst.tex %%%%%%%%%%%%%%%%%%%%%%%%%
%
% This is the LaTeX source for the instructions to authors using
% the LaTeX document class 'llncs.cls' for contributions to
% the Lecture Notes in Computer Sciences series.
% http://www.springer.com/lncs       Springer Heidelberg 2006/05/04
%
% It may be used as a template for your own input - copy it
% to a new file with a new name and use it as the basis
% for your article.
%
% NB: the document class 'llncs' has its own and detailed documentation, see
% ftp://ftp.springer.de/data/pubftp/pub/tex/latex/llncs/latex2e/llncsdoc.pdf
%
%%%%%%%%%%%%%%%%%%%%%%%%%%%%%%%%%%%%%%%%%%%%%%%%%%%%%%%%%%%%%%%%%%%


\documentclass[runningheads,a4paper]{llncs}

\usepackage{amssymb}
\setcounter{tocdepth}{3}
\usepackage{graphicx}
\usepackage{subcaption}
\usepackage{listings}
%\lstset{
%	numbers=left
%	language=Java
%	frame=single,
%	breaklines=true,
	%postbreak=\raisebox{0ex}[0ex][0ex]{\ensuremath{\color{red}\hookrightarrow\space}}
%}
\renewcommand{\lstlistingname}{Code}

\captionsetup{compatibility=false}

\usepackage{url}
\urldef{\mailsa}\path|{alfred.hofmann, ursula.barth, ingrid.haas, frank.holzwarth,|
\urldef{\mailsb}\path|anna.kramer, leonie.kunz, christine.reiss, nicole.sator,|
\urldef{\mailsc}\path|erika.siebert-cole, peter.strasser, lncs}@springer.com|    
\newcommand{\keywords}[1]{\par\addvspace\baselineskip
\noindent\keywordname\enspace\ignorespaces#1}

\begin{document}

\mainmatter  


\newpage
\tableofcontents
\newpage

\abstract{
% what we have done with the dataset and project
	This assignment represents about exploiting WebGraph~\cite{boldi2004webgraph}, one of graph frameworks, for discovering Bitcoin user's graph through some statistic tasks and experiments including degree distributions, finding Strongly Connected Components and a random Bread First Search.
	% These experiments are compared with others on different dataset. 
	%The results show that......
} 

\section{Introduction}
\label{Intro}

% Describe that dataset and generally about sections
	
\section{Block chain}
\label{BC}
% describe and explain steps about preprocessing data 
% discuss about the result

\section{Selfish miner and private block chain}
\label{CA}

% general about the algorithm
% discuss about the result




\section{The simulation}
\label{TS}



\section{Experiments}
\label{Exper}

\section{Conclusion}
\label{Con}
The assignment exploits WebGraph framework to solve some statistics tasks and experiments from the paper~\cite{broder2000graph}. With these works, the framework shows their useful techniques and tools which simply exploits large graphs.

\section{Ackowledgements}
I would like to show my gratitude to Prof. Laura Ricci about lessons.
Finally, I would also like to extend my thank to \LaTeX $ $ and Springer for this format.

\bibliography{Ref}
\bibliographystyle{plain}

\end{document}

\iffals
\begin{figure*}[t!]
	\centering
	\begin{subfigure}[b]{0.6\textwidth}
		%\centering
		\includegraphics[scale = 0.6]{image/Unipi_Image}
		\caption{In-degree distribution with an exponent 4.5}
	\end{subfigure}
	\\
	\begin{subfigure}[b]{0.6\textwidth}
		%\centering
		\includegraphics[scale = 0.6]{image/Unipi_Image}
		\caption{Out-degree distribution with an exponent 5.5}
	\end{subfigure}
	\caption{In-degree and Out-degree distributions}
		\label{Fig:Distri}
\end{figure*}
\textbf{Second Part}:
\begin{figure*}[t!]
	\centering
	\begin{subfigure}[b]{0.6\textwidth}
		%\centering
		\includegraphics[scale = 0.6]{image/Unipi_Image}
		\caption{Size of WCC distribution with an exponent 10 of power law}
	\end{subfigure}
	\\
	\begin{subfigure}[b]{0.6\textwidth}
		%\centering
		\includegraphics[scale = 0.6]{image/Unipi_Image}
		\caption{Size of SCC distribution with an exponent 9 of power law}
	\end{subfigure}

	\caption{Size of WCC and SCC distributions}
		\label{Fig:WCC_SCC}
\end{figure*}

\begin{figure*}[t!]

	\centering
	\begin{subfigure}[b]{0.6\textwidth}
		%\centering
		\includegraphics[scale = 0.6]{image/Unipi_Image}
		\caption{Experiment with BFS on forward direction}
	\end{subfigure}
	\\
	\begin{subfigure}[b]{0.6\textwidth}
		%\centering
		\includegraphics[scale = 0.6]{image/Unipi_Image}
		\caption{Experiment with BFS on backward direction}
	\end{subfigure}
	\\
	\begin{subfigure}[b]{0.6\textwidth}
		%\centering
		\includegraphics[scale = 0.6]{image/Unipi_Image}
		\caption{Experiment with BFS on both directions}
	\end{subfigure}
	\caption{Experiments of BFS algorithm on three kinds of graph}
		\label{Fig:BFS}
\end{figure*}